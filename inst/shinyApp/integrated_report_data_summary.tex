% Options for packages loaded elsewhere
\PassOptionsToPackage{unicode}{hyperref}
\PassOptionsToPackage{hyphens}{url}
%
\documentclass[
]{article}
\usepackage{lmodern}
\usepackage{amssymb,amsmath}
\usepackage{ifxetex,ifluatex}
\ifnum 0\ifxetex 1\fi\ifluatex 1\fi=0 % if pdftex
  \usepackage[T1]{fontenc}
  \usepackage[utf8]{inputenc}
  \usepackage{textcomp} % provide euro and other symbols
\else % if luatex or xetex
  \usepackage{unicode-math}
  \defaultfontfeatures{Scale=MatchLowercase}
  \defaultfontfeatures[\rmfamily]{Ligatures=TeX,Scale=1}
\fi
% Use upquote if available, for straight quotes in verbatim environments
\IfFileExists{upquote.sty}{\usepackage{upquote}}{}
\IfFileExists{microtype.sty}{% use microtype if available
  \usepackage[]{microtype}
  \UseMicrotypeSet[protrusion]{basicmath} % disable protrusion for tt fonts
}{}
\makeatletter
\@ifundefined{KOMAClassName}{% if non-KOMA class
  \IfFileExists{parskip.sty}{%
    \usepackage{parskip}
  }{% else
    \setlength{\parindent}{0pt}
    \setlength{\parskip}{6pt plus 2pt minus 1pt}}
}{% if KOMA class
  \KOMAoptions{parskip=half}}
\makeatother
\usepackage{xcolor}
\IfFileExists{xurl.sty}{\usepackage{xurl}}{} % add URL line breaks if available
\IfFileExists{bookmark.sty}{\usepackage{bookmark}}{\usepackage{hyperref}}
\hypersetup{
  hidelinks,
  pdfcreator={LaTeX via pandoc}}
\urlstyle{same} % disable monospaced font for URLs
\usepackage[margin=1in]{geometry}
\usepackage{color}
\usepackage{fancyvrb}
\newcommand{\VerbBar}{|}
\newcommand{\VERB}{\Verb[commandchars=\\\{\}]}
\DefineVerbatimEnvironment{Highlighting}{Verbatim}{commandchars=\\\{\}}
% Add ',fontsize=\small' for more characters per line
\usepackage{framed}
\definecolor{shadecolor}{RGB}{248,248,248}
\newenvironment{Shaded}{\begin{snugshade}}{\end{snugshade}}
\newcommand{\AlertTok}[1]{\textcolor[rgb]{0.94,0.16,0.16}{#1}}
\newcommand{\AnnotationTok}[1]{\textcolor[rgb]{0.56,0.35,0.01}{\textbf{\textit{#1}}}}
\newcommand{\AttributeTok}[1]{\textcolor[rgb]{0.77,0.63,0.00}{#1}}
\newcommand{\BaseNTok}[1]{\textcolor[rgb]{0.00,0.00,0.81}{#1}}
\newcommand{\BuiltInTok}[1]{#1}
\newcommand{\CharTok}[1]{\textcolor[rgb]{0.31,0.60,0.02}{#1}}
\newcommand{\CommentTok}[1]{\textcolor[rgb]{0.56,0.35,0.01}{\textit{#1}}}
\newcommand{\CommentVarTok}[1]{\textcolor[rgb]{0.56,0.35,0.01}{\textbf{\textit{#1}}}}
\newcommand{\ConstantTok}[1]{\textcolor[rgb]{0.00,0.00,0.00}{#1}}
\newcommand{\ControlFlowTok}[1]{\textcolor[rgb]{0.13,0.29,0.53}{\textbf{#1}}}
\newcommand{\DataTypeTok}[1]{\textcolor[rgb]{0.13,0.29,0.53}{#1}}
\newcommand{\DecValTok}[1]{\textcolor[rgb]{0.00,0.00,0.81}{#1}}
\newcommand{\DocumentationTok}[1]{\textcolor[rgb]{0.56,0.35,0.01}{\textbf{\textit{#1}}}}
\newcommand{\ErrorTok}[1]{\textcolor[rgb]{0.64,0.00,0.00}{\textbf{#1}}}
\newcommand{\ExtensionTok}[1]{#1}
\newcommand{\FloatTok}[1]{\textcolor[rgb]{0.00,0.00,0.81}{#1}}
\newcommand{\FunctionTok}[1]{\textcolor[rgb]{0.00,0.00,0.00}{#1}}
\newcommand{\ImportTok}[1]{#1}
\newcommand{\InformationTok}[1]{\textcolor[rgb]{0.56,0.35,0.01}{\textbf{\textit{#1}}}}
\newcommand{\KeywordTok}[1]{\textcolor[rgb]{0.13,0.29,0.53}{\textbf{#1}}}
\newcommand{\NormalTok}[1]{#1}
\newcommand{\OperatorTok}[1]{\textcolor[rgb]{0.81,0.36,0.00}{\textbf{#1}}}
\newcommand{\OtherTok}[1]{\textcolor[rgb]{0.56,0.35,0.01}{#1}}
\newcommand{\PreprocessorTok}[1]{\textcolor[rgb]{0.56,0.35,0.01}{\textit{#1}}}
\newcommand{\RegionMarkerTok}[1]{#1}
\newcommand{\SpecialCharTok}[1]{\textcolor[rgb]{0.00,0.00,0.00}{#1}}
\newcommand{\SpecialStringTok}[1]{\textcolor[rgb]{0.31,0.60,0.02}{#1}}
\newcommand{\StringTok}[1]{\textcolor[rgb]{0.31,0.60,0.02}{#1}}
\newcommand{\VariableTok}[1]{\textcolor[rgb]{0.00,0.00,0.00}{#1}}
\newcommand{\VerbatimStringTok}[1]{\textcolor[rgb]{0.31,0.60,0.02}{#1}}
\newcommand{\WarningTok}[1]{\textcolor[rgb]{0.56,0.35,0.01}{\textbf{\textit{#1}}}}
\usepackage{longtable,booktabs}
% Correct order of tables after \paragraph or \subparagraph
\usepackage{etoolbox}
\makeatletter
\patchcmd\longtable{\par}{\if@noskipsec\mbox{}\fi\par}{}{}
\makeatother
% Allow footnotes in longtable head/foot
\IfFileExists{footnotehyper.sty}{\usepackage{footnotehyper}}{\usepackage{footnote}}
\makesavenoteenv{longtable}
\usepackage{graphicx,grffile}
\makeatletter
\def\maxwidth{\ifdim\Gin@nat@width>\linewidth\linewidth\else\Gin@nat@width\fi}
\def\maxheight{\ifdim\Gin@nat@height>\textheight\textheight\else\Gin@nat@height\fi}
\makeatother
% Scale images if necessary, so that they will not overflow the page
% margins by default, and it is still possible to overwrite the defaults
% using explicit options in \includegraphics[width, height, ...]{}
\setkeys{Gin}{width=\maxwidth,height=\maxheight,keepaspectratio}
% Set default figure placement to htbp
\makeatletter
\def\fps@figure{htbp}
\makeatother
\setlength{\emergencystretch}{3em} % prevent overfull lines
\providecommand{\tightlist}{%
  \setlength{\itemsep}{0pt}\setlength{\parskip}{0pt}}
\setcounter{secnumdepth}{-\maxdimen} % remove section numbering

\author{}
\date{\vspace{-2.5em}}

\begin{document}

\hypertarget{intro}{%
\section{Introduction}\label{intro}}

This is a document that outlines a vignette for implementing privacy
preserving survival models, meta-analyzing hazard ratios in the
DataSHIELD platform and performing data documentation.

We used the \textbf{bookdown} package {[}@R-bookdown{]}, R Markdown and
\textbf{knitr} {[}@xie2015{]} for this document. Our package
\textbf{dsSurvival}
{[}@Banerjeef{]}{[}@soumya\_banerjee\_2021\_4917552{]} uses the
\textbf{metafor} package for meta-analysis {[}@Viechtbauer2010{]}.

\hypertarget{survival-models}{%
\subsection{Survival models}\label{survival-models}}

Survival models are used extensively in healthcare. Previously building
survival models in DataSHIELD involved building piecewise exponential
regression models. This is an approximation and involves having to
define appropriate time buckets. A lack of familiarity with this
approach also makes people suspicious.

The scope of our package implementation is restricted to being
study-level meta-analysis (SLMA) rather than full likelihood.

\hypertarget{computational-workflow}{%
\section{Computational workflow}\label{computational-workflow}}

The computational steps are outlined below. The first step is connecting
to the server and loading the survival data.

\begin{Shaded}
\begin{Highlighting}[]
\KeywordTok{library}\NormalTok{(knitr)}
\KeywordTok{library}\NormalTok{(rmarkdown)}
\KeywordTok{library}\NormalTok{(tinytex)}
\KeywordTok{library}\NormalTok{(survival)}
\KeywordTok{library}\NormalTok{(metafor)}
\KeywordTok{library}\NormalTok{(ggplot2)}
\KeywordTok{library}\NormalTok{(dsSurvivalClient)}
\KeywordTok{require}\NormalTok{(}\StringTok{'DSI'}\NormalTok{)}
\KeywordTok{require}\NormalTok{(}\StringTok{'DSOpal'}\NormalTok{)}
\KeywordTok{require}\NormalTok{(}\StringTok{'dsBaseClient'}\NormalTok{)}

\NormalTok{builder <-}\StringTok{ }\NormalTok{DSI}\OperatorTok{::}\KeywordTok{newDSLoginBuilder}\NormalTok{()}

\NormalTok{builder}\OperatorTok{$}\KeywordTok{append}\NormalTok{(}\DataTypeTok{server=}\StringTok{"server1"}\NormalTok{, }\DataTypeTok{url=}\StringTok{"https://opal-sandbox.mrc-epid.cam.ac.uk"}\NormalTok{,}
                \DataTypeTok{user=}\StringTok{"dsuser"}\NormalTok{, }\DataTypeTok{password=}\StringTok{"password"}\NormalTok{, }
               \DataTypeTok{table =} \StringTok{"SURVIVAL.EXPAND_NO_MISSING1"}\NormalTok{)}

\NormalTok{builder}\OperatorTok{$}\KeywordTok{append}\NormalTok{(}\DataTypeTok{server=}\StringTok{"server2"}\NormalTok{, }\DataTypeTok{url=}\StringTok{"https://opal-sandbox.mrc-epid.cam.ac.uk"}\NormalTok{,}
               \DataTypeTok{user=}\StringTok{"dsuser"}\NormalTok{, }\DataTypeTok{password=}\StringTok{"password"}\NormalTok{, }
               \DataTypeTok{table =} \StringTok{"SURVIVAL.EXPAND_NO_MISSING2"}\NormalTok{)}

\NormalTok{builder}\OperatorTok{$}\KeywordTok{append}\NormalTok{(}\DataTypeTok{server=}\StringTok{"server3"}\NormalTok{, }\DataTypeTok{url=}\StringTok{"https://opal-sandbox.mrc-epid.cam.ac.uk"}\NormalTok{,}
               \DataTypeTok{user=}\StringTok{"dsuser"}\NormalTok{, }\DataTypeTok{password=}\StringTok{"password"}\NormalTok{, }
               \DataTypeTok{table =} \StringTok{"SURVIVAL.EXPAND_NO_MISSING3"}\NormalTok{)          }

\NormalTok{logindata <-}\StringTok{ }\NormalTok{builder}\OperatorTok{$}\KeywordTok{build}\NormalTok{()}

\NormalTok{connections <-}\StringTok{ }\NormalTok{DSI}\OperatorTok{::}\KeywordTok{datashield.login}\NormalTok{(}\DataTypeTok{logins =}\NormalTok{ logindata, }\DataTypeTok{assign =} \OtherTok{TRUE}\NormalTok{, }\DataTypeTok{symbol =} \StringTok{"D"}\NormalTok{) }
\end{Highlighting}
\end{Shaded}

\hypertarget{creating-server-side-variables-for-survival-analysis}{%
\subsection{Creating server-side variables for survival
analysis}\label{creating-server-side-variables-for-survival-analysis}}

We now outline the steps for analysing survival data.

We show this using synthetic data. There are 3 data sets that are held
on the same server but can be considered to be on separate
servers/sites.

The \textbf{cens} variable has the event information and the
\textbf{survtime} variable has the time information. There is also age
and gender information in the variables named \textbf{age} and
\textbf{female}, respectively.

We will look at how age and gender affect survival time and then
meta-analyze the hazard ratios from the survival model.

\begin{itemize}
\tightlist
\item
  make sure that the outcome variable is numeric
\end{itemize}

\begin{Shaded}
\begin{Highlighting}[]
\KeywordTok{ds.asNumeric}\NormalTok{(}\DataTypeTok{x.name =} \StringTok{"D$cens"}\NormalTok{,}
             \DataTypeTok{newobj =} \StringTok{"EVENT"}\NormalTok{,}
             \DataTypeTok{datasources =}\NormalTok{ connections)}
         
\KeywordTok{ds.asNumeric}\NormalTok{(}\DataTypeTok{x.name =} \StringTok{"D$survtime"}\NormalTok{,}
             \DataTypeTok{newobj =} \StringTok{"SURVTIME"}\NormalTok{,}
             \DataTypeTok{datasources =}\NormalTok{ connections)}
\end{Highlighting}
\end{Shaded}

\begin{itemize}
\tightlist
\item
  convert time id variable to a factor
\end{itemize}

\begin{Shaded}
\begin{Highlighting}[]
\KeywordTok{ds.asFactor}\NormalTok{(}\DataTypeTok{input.var.name =} \StringTok{"D$time.id"}\NormalTok{,}
            \DataTypeTok{newobj =} \StringTok{"TID"}\NormalTok{,}
            \DataTypeTok{datasources =}\NormalTok{ connections)}
\end{Highlighting}
\end{Shaded}

\begin{itemize}
\tightlist
\item
  create in the server-side the log(survtime) variable
\end{itemize}

\begin{Shaded}
\begin{Highlighting}[]
\KeywordTok{ds.log}\NormalTok{(}\DataTypeTok{x =} \StringTok{"D$survtime"}\NormalTok{,}
       \DataTypeTok{newobj =} \StringTok{"log.surv"}\NormalTok{,}
       \DataTypeTok{datasources =}\NormalTok{ connections)}
\end{Highlighting}
\end{Shaded}

\begin{itemize}
\tightlist
\item
  create start time variable
\end{itemize}

\begin{Shaded}
\begin{Highlighting}[]
\KeywordTok{ds.asNumeric}\NormalTok{(}\DataTypeTok{x.name =} \StringTok{"D$starttime"}\NormalTok{,}
             \DataTypeTok{newobj =} \StringTok{"STARTTIME"}\NormalTok{,}
             \DataTypeTok{datasources =}\NormalTok{ connections)}
\KeywordTok{ds.asNumeric}\NormalTok{(}\DataTypeTok{x.name =} \StringTok{"D$endtime"}\NormalTok{,}
             \DataTypeTok{newobj =} \StringTok{"ENDTIME"}\NormalTok{,}
             \DataTypeTok{datasources =}\NormalTok{ connections)}
\end{Highlighting}
\end{Shaded}

\hypertarget{create-survival-object-and-call-ds.coxph.slma}{%
\subsection{Create survival object and call
ds.coxph.SLMA()}\label{create-survival-object-and-call-ds.coxph.slma}}

There are two options to generate the survival object. You can generate
it separately or in line.

If a survival object is generated separately, it is stored on the server
and can be used later in an assign function ( ds.coxphSLMAassign() ).
The motivation for creating the model on the server side is inspired
from the ds.glmassign functions. This allows the survival model to be
stored on the server and can be used later for diagnostics.

\begin{itemize}
\tightlist
\item
  use constructed Surv object in \emph{ds.coxph.SLMA()}
\end{itemize}

\begin{Shaded}
\begin{Highlighting}[]
\NormalTok{dsSurvivalClient}\OperatorTok{::}\KeywordTok{ds.Surv}\NormalTok{(}\DataTypeTok{time=}\StringTok{'STARTTIME'}\NormalTok{, }\DataTypeTok{time2=}\StringTok{'ENDTIME'}\NormalTok{, }
                      \DataTypeTok{event =} \StringTok{'EVENT'}\NormalTok{, }\DataTypeTok{objectname=}\StringTok{'surv_object'}\NormalTok{,}
                      \DataTypeTok{type=}\StringTok{'counting'}\NormalTok{)}
              
\NormalTok{coxph_model_full <-}\StringTok{ }\NormalTok{dsSurvivalClient}\OperatorTok{::}\KeywordTok{ds.coxph.SLMA}\NormalTok{(}\DataTypeTok{formula =} \StringTok{'surv_object~D$age+D$female'}\NormalTok{)}
\end{Highlighting}
\end{Shaded}

\begin{itemize}
\tightlist
\item
  use direct inline call to \emph{survival::Surv()}
\end{itemize}

\begin{Shaded}
\begin{Highlighting}[]
\NormalTok{dsSurvivalClient}\OperatorTok{::}\KeywordTok{ds.coxph.SLMA}\NormalTok{(}\DataTypeTok{formula =} \StringTok{'survival::Surv(time=SURVTIME,event=EVENT)~D$age+D$female'}\NormalTok{, }
                                \DataTypeTok{dataName =} \StringTok{'D'}\NormalTok{, }
                                \DataTypeTok{datasources =}\NormalTok{ connections)}
\end{Highlighting}
\end{Shaded}

\begin{itemize}
\tightlist
\item
  call with \emph{survival::strata()}
\end{itemize}

The strata() option allows us to relax some of the proportional hazards
assumptions. It allows fitting of a separate baseline hazard function
within each strata.

\begin{Shaded}
\begin{Highlighting}[]
\NormalTok{coxph_model_strata <-}\StringTok{ }\NormalTok{dsSurvivalClient}\OperatorTok{::}\KeywordTok{ds.coxph.SLMA}\NormalTok{(}\DataTypeTok{formula =} \StringTok{'surv_object~D$age + }
\StringTok{                          survival::strata(D$female)'}\NormalTok{)}
\KeywordTok{summary}\NormalTok{(coxph_model_strata)}
\end{Highlighting}
\end{Shaded}

\hypertarget{summary-of-survival-objects}{%
\subsection{Summary of survival
objects}\label{summary-of-survival-objects}}

We can also summarize a server-side object of type
\emph{survival::Surv()} using a call to \emph{ds.coxphSummary()}. This
will provide a non-disclosive summary of the server-side object. The
server-side survival object can be created using ds.coxphSLMAassign().
An example call is shown below:

\begin{Shaded}
\begin{Highlighting}[]
\NormalTok{dsSurvivalClient}\OperatorTok{::}\KeywordTok{ds.coxphSummary}\NormalTok{(}\DataTypeTok{x =} \StringTok{'coxph_serverside'}\NormalTok{)}
\end{Highlighting}
\end{Shaded}

\hypertarget{diagnostics-for-cox-proportional-hazards-models}{%
\subsection{Diagnostics for Cox proportional hazards
models}\label{diagnostics-for-cox-proportional-hazards-models}}

We have also created functions to test for the assumptions of Cox
proportional hazards models. This requires a call to the function
ds.cox.zphSLMA. Before the call, a server-side object has to be created
using the assign function ds.coxphSLMAassign().

All the function calls are shown below:

\begin{Shaded}
\begin{Highlighting}[]
\NormalTok{dsSurvivalClient}\OperatorTok{::}\KeywordTok{ds.coxphSLMAassign}\NormalTok{(}\DataTypeTok{formula =} \StringTok{'surv_object~D$age+D$female'}\NormalTok{,}
                            \DataTypeTok{objectname =} \StringTok{'coxph_serverside'}\NormalTok{)}
                
\NormalTok{dsSurvivalClient}\OperatorTok{::}\KeywordTok{ds.cox.zphSLMA}\NormalTok{(}\DataTypeTok{fit =} \StringTok{'coxph_serverside'}\NormalTok{)}

\NormalTok{dsSurvivalClient}\OperatorTok{::}\KeywordTok{ds.coxphSummary}\NormalTok{(}\DataTypeTok{x =} \StringTok{'coxph_serverside'}\NormalTok{)}
\end{Highlighting}
\end{Shaded}

These diagnostics can allow an analyst to determine if the proportional
hazards assumption in Cox proportional hazards models is satisfied. If
the p-values shown below are greater than 0.05 for any co-variate, then
the proportional hazards assumption is correct for that co-variate.

If the proportional hazards assumptions are violated (p-values less than
0.05), then the analyst will have to modify the model. Modifications may
include introducing strata or using time-dependent covariates. Please
see the links below for more information on this:

\begin{itemize}
\item
  \url{https://stats.stackexchange.com/questions/317336/interpreting-r-coxph-cox-zph}
\item
  \url{https://stats.stackexchange.com/questions/144923/extended-cox-model-and-cox-zph/238964\#238964}
\end{itemize}

A diagnostic summary is shown below.

\begin{verbatim}
## surv_object~D$age+D$female
\end{verbatim}

\begin{verbatim}
## NULL
\end{verbatim}

\begin{verbatim}
## $study1
##          chisq df    p
## D$age    1.022  1 0.31
## D$female 0.364  1 0.55
## GLOBAL   1.239  2 0.54
## 
## $study2
##              chisq df    p
## D$age    -1389.472  1 1.00
## D$female     0.591  1 0.44
## GLOBAL    -857.492  2 1.00
## 
## $study3
##          chisq df       p
## D$age    15.27  1 9.3e-05
## D$female  8.04  1  0.0046
## GLOBAL   23.31  2 8.7e-06
\end{verbatim}

\begin{verbatim}
## $study1
## Call:
## survival::coxph(formula = formula, data = dataTable, weights = weights, 
##     ties = ties, singular.ok = singular.ok, model = model, x = x, 
##     y = y)
## 
##   n= 2060, number of events= 426 
## 
##                coef exp(coef)  se(coef)      z Pr(>|z|)    
## D$age      0.041609  1.042487  0.003498 11.894  < 2e-16 ***
## D$female1 -0.660002  0.516850  0.099481 -6.634 3.26e-11 ***
## ---
## Signif. codes:  0 '***' 0.001 '**' 0.01 '*' 0.05 '.' 0.1 ' ' 1
## 
##           exp(coef) exp(-coef) lower .95 upper .95
## D$age        1.0425     0.9592    1.0354    1.0497
## D$female1    0.5169     1.9348    0.4253    0.6281
## 
## Concordance= 0.676  (se = 0.014 )
## Likelihood ratio test= 170.7  on 2 df,   p=<2e-16
## Wald test            = 168.2  on 2 df,   p=<2e-16
## Score (logrank) test = 166.3  on 2 df,   p=<2e-16
## 
## 
## $study2
## Call:
## survival::coxph(formula = formula, data = dataTable, weights = weights, 
##     ties = ties, singular.ok = singular.ok, model = model, x = x, 
##     y = y)
## 
##   n= 1640, number of events= 300 
## 
##               coef exp(coef) se(coef)      z Pr(>|z|)    
## D$age      0.04067   1.04151  0.00416  9.776  < 2e-16 ***
## D$female1 -0.62756   0.53389  0.11767 -5.333 9.66e-08 ***
## ---
## Signif. codes:  0 '***' 0.001 '**' 0.01 '*' 0.05 '.' 0.1 ' ' 1
## 
##           exp(coef) exp(-coef) lower .95 upper .95
## D$age        1.0415     0.9601    1.0331    1.0500
## D$female1    0.5339     1.8730    0.4239    0.6724
## 
## Concordance= 0.674  (se = 0.017 )
## Likelihood ratio test= 117.8  on 2 df,   p=<2e-16
## Wald test            = 115.2  on 2 df,   p=<2e-16
## Score (logrank) test = 116.4  on 2 df,   p=<2e-16
## 
## 
## $study3
## Call:
## survival::coxph(formula = formula, data = dataTable, weights = weights, 
##     ties = ties, singular.ok = singular.ok, model = model, x = x, 
##     y = y)
## 
##   n= 2688, number of events= 578 
## 
##                coef exp(coef)  se(coef)      z Pr(>|z|)    
## D$age      0.042145  1.043045  0.003086 13.655  < 2e-16 ***
## D$female1 -0.599238  0.549230  0.084305 -7.108 1.18e-12 ***
## ---
## Signif. codes:  0 '***' 0.001 '**' 0.01 '*' 0.05 '.' 0.1 ' ' 1
## 
##           exp(coef) exp(-coef) lower .95 upper .95
## D$age        1.0430     0.9587    1.0368    1.0494
## D$female1    0.5492     1.8207    0.4656    0.6479
## 
## Concordance= 0.699  (se = 0.011 )
## Likelihood ratio test= 227.9  on 2 df,   p=<2e-16
## Wald test            = 228.4  on 2 df,   p=<2e-16
## Score (logrank) test = 229.4  on 2 df,   p=<2e-16
\end{verbatim}

\hypertarget{meta-analyze-hazard-ratios}{%
\subsection{Meta-analyze hazard
ratios}\label{meta-analyze-hazard-ratios}}

We now outline how the hazard ratios from the survival models are
meta-analyzed. We use the \emph{metafor} package for meta-analysis. We
show the summary of an example meta-analysis and a forest plot below.
The forest plot shows a basic example of meta-analyzed hazard ratios
from a survival model (analyzed in dsSurvivalClient).

The log-hazard ratios and their standard errors from each study can be
found after running \emph{ds.coxphSLMA()}

The hazard ratios can then be meta-analyzed by running the commands
shown below. These commands get the hazard ratios correspondng to age in
the survival model.

\begin{Shaded}
\begin{Highlighting}[]
\NormalTok{input_logHR =}\StringTok{ }\KeywordTok{c}\NormalTok{(coxph_model_full}\OperatorTok{$}\NormalTok{study1}\OperatorTok{$}\NormalTok{coefficients[}\DecValTok{1}\NormalTok{,}\DecValTok{2}\NormalTok{], }
\NormalTok{        coxph_model_full}\OperatorTok{$}\NormalTok{study2}\OperatorTok{$}\NormalTok{coefficients[}\DecValTok{1}\NormalTok{,}\DecValTok{2}\NormalTok{], }
\NormalTok{        coxph_model_full}\OperatorTok{$}\NormalTok{study3}\OperatorTok{$}\NormalTok{coefficients[}\DecValTok{1}\NormalTok{,}\DecValTok{2}\NormalTok{])}
        
\NormalTok{input_se    =}\StringTok{ }\KeywordTok{c}\NormalTok{(coxph_model_full}\OperatorTok{$}\NormalTok{study1}\OperatorTok{$}\NormalTok{coefficients[}\DecValTok{1}\NormalTok{,}\DecValTok{3}\NormalTok{], }
\NormalTok{        coxph_model_full}\OperatorTok{$}\NormalTok{study2}\OperatorTok{$}\NormalTok{coefficients[}\DecValTok{1}\NormalTok{,}\DecValTok{3}\NormalTok{], }
\NormalTok{        coxph_model_full}\OperatorTok{$}\NormalTok{study3}\OperatorTok{$}\NormalTok{coefficients[}\DecValTok{1}\NormalTok{,}\DecValTok{3}\NormalTok{])}
        
\NormalTok{metafor}\OperatorTok{::}\KeywordTok{rma}\NormalTok{(input_logHR, }\DataTypeTok{sei =}\NormalTok{ input_se, }\DataTypeTok{method =} \StringTok{'REML'}\NormalTok{)}
\end{Highlighting}
\end{Shaded}

A summary of this meta-analyzed model is shown below.

\begin{verbatim}
## 
## Random-Effects Model (k = 3; tau^2 estimator: REML)
## 
##   logLik  deviance       AIC       BIC      AICc 
##   9.3824  -18.7648  -14.7648  -17.3785   -2.7648   
## 
## tau^2 (estimated amount of total heterogeneity): 0 (SE = 0.0000)
## tau (square root of estimated tau^2 value):      0
## I^2 (total heterogeneity / total variability):   0.00%
## H^2 (total variability / sampling variability):  1.00
## 
## Test for Heterogeneity:
## Q(df = 2) = 0.0880, p-val = 0.9569
## 
## Model Results:
## 
## estimate      se      zval    pval   ci.lb   ci.ub 
##   1.0425  0.0020  515.4456  <.0001  1.0385  1.0465  *** 
## 
## ---
## Signif. codes:  0 '***' 0.001 '**' 0.01 '*' 0.05 '.' 0.1 ' ' 1
\end{verbatim}

We now show a forest plot with the meta-analyzed hazard ratios. The
hazard ratios come from the dsSurvivalClient function
\emph{ds.coxphSLMA()}. The plot shows the coefficients for age in the
survival model. The command is shown below.

\begin{Shaded}
\begin{Highlighting}[]
\NormalTok{metafor}\OperatorTok{::}\KeywordTok{forest.rma}\NormalTok{(}\DataTypeTok{x =}\NormalTok{ meta_model, }\DataTypeTok{digits =} \DecValTok{4}\NormalTok{) }
\end{Highlighting}
\end{Shaded}

\begin{figure}
\centering
\includegraphics{integrated_report_data_summary_files/figure-latex/unnamed-chunk-19-1.pdf}
\caption{Example forest plot of meta-analyzed hazard ratios.}
\end{figure}

\hypertarget{data-documentation}{%
\subsection{Data documentation}\label{data-documentation}}

\hypertarget{descriptive-statistics-of-data}{%
\subsubsection{Descriptive statistics of
data}\label{descriptive-statistics-of-data}}

We show below how to generate descriptive summary statistics of data.
Knowing this helps the analyst understand the characteristics of the
data better. The analyst can then perform data transformations (if
necessary) and build the appropriate survival models.

\begin{Shaded}
\begin{Highlighting}[]
\KeywordTok{library}\NormalTok{(dsHelper)}

\NormalTok{list_stats_summary <-}\StringTok{ }\NormalTok{dsHelper}\OperatorTok{::}\KeywordTok{dh.getStats}\NormalTok{(}\DataTypeTok{conns =}\NormalTok{ connections, }
                                            \DataTypeTok{df =} \StringTok{'D'}\NormalTok{, }
                        \DataTypeTok{vars =} \KeywordTok{c}\NormalTok{(}\StringTok{'age.60'}\NormalTok{, }\StringTok{'female'}\NormalTok{, }\StringTok{'starttime'}\NormalTok{, }\StringTok{'endtime'}\NormalTok{) }
\NormalTok{                        ) }
\end{Highlighting}
\end{Shaded}

\begin{Shaded}
\begin{Highlighting}[]
\NormalTok{knitr}\OperatorTok{::}\KeywordTok{kable}\NormalTok{(}\KeywordTok{as.data.frame}\NormalTok{(list_stats_summary}\OperatorTok{$}\NormalTok{categorical))}
\NormalTok{knitr}\OperatorTok{::}\KeywordTok{kable}\NormalTok{(}\KeywordTok{as.data.frame}\NormalTok{(list_stats_summary}\OperatorTok{$}\NormalTok{continuous))}
\end{Highlighting}
\end{Shaded}

\begin{longtable}[]{@{}llrlrrrrr@{}}
\toprule
\begin{minipage}[b]{0.08\columnwidth}\raggedright
variable\strut
\end{minipage} & \begin{minipage}[b]{0.08\columnwidth}\raggedright
category\strut
\end{minipage} & \begin{minipage}[b]{0.05\columnwidth}\raggedleft
value\strut
\end{minipage} & \begin{minipage}[b]{0.08\columnwidth}\raggedright
cohort\strut
\end{minipage} & \begin{minipage}[b]{0.08\columnwidth}\raggedleft
cohort\_n\strut
\end{minipage} & \begin{minipage}[b]{0.07\columnwidth}\raggedleft
valid\_n\strut
\end{minipage} & \begin{minipage}[b]{0.09\columnwidth}\raggedleft
missing\_n\strut
\end{minipage} & \begin{minipage}[b]{0.10\columnwidth}\raggedleft
valid\_perc\strut
\end{minipage} & \begin{minipage}[b]{0.12\columnwidth}\raggedleft
missing\_perc\strut
\end{minipage}\tabularnewline
\midrule
\endhead
\begin{minipage}[t]{0.08\columnwidth}\raggedright
female\strut
\end{minipage} & \begin{minipage}[t]{0.08\columnwidth}\raggedright
0\strut
\end{minipage} & \begin{minipage}[t]{0.05\columnwidth}\raggedleft
942\strut
\end{minipage} & \begin{minipage}[t]{0.08\columnwidth}\raggedright
study1\strut
\end{minipage} & \begin{minipage}[t]{0.08\columnwidth}\raggedleft
2060\strut
\end{minipage} & \begin{minipage}[t]{0.07\columnwidth}\raggedleft
2060\strut
\end{minipage} & \begin{minipage}[t]{0.09\columnwidth}\raggedleft
0\strut
\end{minipage} & \begin{minipage}[t]{0.10\columnwidth}\raggedleft
45.73\strut
\end{minipage} & \begin{minipage}[t]{0.12\columnwidth}\raggedleft
0\strut
\end{minipage}\tabularnewline
\begin{minipage}[t]{0.08\columnwidth}\raggedright
female\strut
\end{minipage} & \begin{minipage}[t]{0.08\columnwidth}\raggedright
1\strut
\end{minipage} & \begin{minipage}[t]{0.05\columnwidth}\raggedleft
1118\strut
\end{minipage} & \begin{minipage}[t]{0.08\columnwidth}\raggedright
study1\strut
\end{minipage} & \begin{minipage}[t]{0.08\columnwidth}\raggedleft
2060\strut
\end{minipage} & \begin{minipage}[t]{0.07\columnwidth}\raggedleft
2060\strut
\end{minipage} & \begin{minipage}[t]{0.09\columnwidth}\raggedleft
0\strut
\end{minipage} & \begin{minipage}[t]{0.10\columnwidth}\raggedleft
54.27\strut
\end{minipage} & \begin{minipage}[t]{0.12\columnwidth}\raggedleft
0\strut
\end{minipage}\tabularnewline
\begin{minipage}[t]{0.08\columnwidth}\raggedright
female\strut
\end{minipage} & \begin{minipage}[t]{0.08\columnwidth}\raggedright
0\strut
\end{minipage} & \begin{minipage}[t]{0.05\columnwidth}\raggedleft
704\strut
\end{minipage} & \begin{minipage}[t]{0.08\columnwidth}\raggedright
study2\strut
\end{minipage} & \begin{minipage}[t]{0.08\columnwidth}\raggedleft
1640\strut
\end{minipage} & \begin{minipage}[t]{0.07\columnwidth}\raggedleft
1640\strut
\end{minipage} & \begin{minipage}[t]{0.09\columnwidth}\raggedleft
0\strut
\end{minipage} & \begin{minipage}[t]{0.10\columnwidth}\raggedleft
42.93\strut
\end{minipage} & \begin{minipage}[t]{0.12\columnwidth}\raggedleft
0\strut
\end{minipage}\tabularnewline
\begin{minipage}[t]{0.08\columnwidth}\raggedright
female\strut
\end{minipage} & \begin{minipage}[t]{0.08\columnwidth}\raggedright
1\strut
\end{minipage} & \begin{minipage}[t]{0.05\columnwidth}\raggedleft
936\strut
\end{minipage} & \begin{minipage}[t]{0.08\columnwidth}\raggedright
study2\strut
\end{minipage} & \begin{minipage}[t]{0.08\columnwidth}\raggedleft
1640\strut
\end{minipage} & \begin{minipage}[t]{0.07\columnwidth}\raggedleft
1640\strut
\end{minipage} & \begin{minipage}[t]{0.09\columnwidth}\raggedleft
0\strut
\end{minipage} & \begin{minipage}[t]{0.10\columnwidth}\raggedleft
57.07\strut
\end{minipage} & \begin{minipage}[t]{0.12\columnwidth}\raggedleft
0\strut
\end{minipage}\tabularnewline
\begin{minipage}[t]{0.08\columnwidth}\raggedright
female\strut
\end{minipage} & \begin{minipage}[t]{0.08\columnwidth}\raggedright
0\strut
\end{minipage} & \begin{minipage}[t]{0.05\columnwidth}\raggedleft
1185\strut
\end{minipage} & \begin{minipage}[t]{0.08\columnwidth}\raggedright
study3\strut
\end{minipage} & \begin{minipage}[t]{0.08\columnwidth}\raggedleft
2688\strut
\end{minipage} & \begin{minipage}[t]{0.07\columnwidth}\raggedleft
2688\strut
\end{minipage} & \begin{minipage}[t]{0.09\columnwidth}\raggedleft
0\strut
\end{minipage} & \begin{minipage}[t]{0.10\columnwidth}\raggedleft
44.08\strut
\end{minipage} & \begin{minipage}[t]{0.12\columnwidth}\raggedleft
0\strut
\end{minipage}\tabularnewline
\begin{minipage}[t]{0.08\columnwidth}\raggedright
female\strut
\end{minipage} & \begin{minipage}[t]{0.08\columnwidth}\raggedright
1\strut
\end{minipage} & \begin{minipage}[t]{0.05\columnwidth}\raggedleft
1503\strut
\end{minipage} & \begin{minipage}[t]{0.08\columnwidth}\raggedright
study3\strut
\end{minipage} & \begin{minipage}[t]{0.08\columnwidth}\raggedleft
2688\strut
\end{minipage} & \begin{minipage}[t]{0.07\columnwidth}\raggedleft
2688\strut
\end{minipage} & \begin{minipage}[t]{0.09\columnwidth}\raggedleft
0\strut
\end{minipage} & \begin{minipage}[t]{0.10\columnwidth}\raggedleft
55.92\strut
\end{minipage} & \begin{minipage}[t]{0.12\columnwidth}\raggedleft
0\strut
\end{minipage}\tabularnewline
\begin{minipage}[t]{0.08\columnwidth}\raggedright
female\strut
\end{minipage} & \begin{minipage}[t]{0.08\columnwidth}\raggedright
0\strut
\end{minipage} & \begin{minipage}[t]{0.05\columnwidth}\raggedleft
2831\strut
\end{minipage} & \begin{minipage}[t]{0.08\columnwidth}\raggedright
combined\strut
\end{minipage} & \begin{minipage}[t]{0.08\columnwidth}\raggedleft
6388\strut
\end{minipage} & \begin{minipage}[t]{0.07\columnwidth}\raggedleft
6388\strut
\end{minipage} & \begin{minipage}[t]{0.09\columnwidth}\raggedleft
0\strut
\end{minipage} & \begin{minipage}[t]{0.10\columnwidth}\raggedleft
44.32\strut
\end{minipage} & \begin{minipage}[t]{0.12\columnwidth}\raggedleft
0\strut
\end{minipage}\tabularnewline
\begin{minipage}[t]{0.08\columnwidth}\raggedright
female\strut
\end{minipage} & \begin{minipage}[t]{0.08\columnwidth}\raggedright
1\strut
\end{minipage} & \begin{minipage}[t]{0.05\columnwidth}\raggedleft
3557\strut
\end{minipage} & \begin{minipage}[t]{0.08\columnwidth}\raggedright
combined\strut
\end{minipage} & \begin{minipage}[t]{0.08\columnwidth}\raggedleft
6388\strut
\end{minipage} & \begin{minipage}[t]{0.07\columnwidth}\raggedleft
6388\strut
\end{minipage} & \begin{minipage}[t]{0.09\columnwidth}\raggedleft
0\strut
\end{minipage} & \begin{minipage}[t]{0.10\columnwidth}\raggedleft
55.68\strut
\end{minipage} & \begin{minipage}[t]{0.12\columnwidth}\raggedleft
0\strut
\end{minipage}\tabularnewline
\bottomrule
\end{longtable}

\begin{longtable}[]{@{}llrrrrrrrrr@{}}
\toprule
\begin{minipage}[b]{0.07\columnwidth}\raggedright
cohort\strut
\end{minipage} & \begin{minipage}[b]{0.07\columnwidth}\raggedright
variable\strut
\end{minipage} & \begin{minipage}[b]{0.04\columnwidth}\raggedleft
mean\strut
\end{minipage} & \begin{minipage}[b]{0.05\columnwidth}\raggedleft
perc\_5\strut
\end{minipage} & \begin{minipage}[b]{0.06\columnwidth}\raggedleft
perc\_50\strut
\end{minipage} & \begin{minipage}[b]{0.06\columnwidth}\raggedleft
perc\_95\strut
\end{minipage} & \begin{minipage}[b]{0.07\columnwidth}\raggedleft
std.dev\strut
\end{minipage} & \begin{minipage}[b]{0.06\columnwidth}\raggedleft
valid\_n\strut
\end{minipage} & \begin{minipage}[b]{0.07\columnwidth}\raggedleft
cohort\_n\strut
\end{minipage} & \begin{minipage}[b]{0.07\columnwidth}\raggedleft
missing\_n\strut
\end{minipage} & \begin{minipage}[b]{0.10\columnwidth}\raggedleft
missing\_perc\strut
\end{minipage}\tabularnewline
\midrule
\endhead
\begin{minipage}[t]{0.07\columnwidth}\raggedright
study1\strut
\end{minipage} & \begin{minipage}[t]{0.07\columnwidth}\raggedright
age.60\strut
\end{minipage} & \begin{minipage}[t]{0.04\columnwidth}\raggedleft
-3.16\strut
\end{minipage} & \begin{minipage}[t]{0.05\columnwidth}\raggedleft
-27.00\strut
\end{minipage} & \begin{minipage}[t]{0.06\columnwidth}\raggedleft
-3.00\strut
\end{minipage} & \begin{minipage}[t]{0.06\columnwidth}\raggedleft
20.00\strut
\end{minipage} & \begin{minipage}[t]{0.07\columnwidth}\raggedleft
14.33\strut
\end{minipage} & \begin{minipage}[t]{0.06\columnwidth}\raggedleft
2060\strut
\end{minipage} & \begin{minipage}[t]{0.07\columnwidth}\raggedleft
2060\strut
\end{minipage} & \begin{minipage}[t]{0.07\columnwidth}\raggedleft
0\strut
\end{minipage} & \begin{minipage}[t]{0.10\columnwidth}\raggedleft
0\strut
\end{minipage}\tabularnewline
\begin{minipage}[t]{0.07\columnwidth}\raggedright
study2\strut
\end{minipage} & \begin{minipage}[t]{0.07\columnwidth}\raggedright
age.60\strut
\end{minipage} & \begin{minipage}[t]{0.04\columnwidth}\raggedleft
-4.01\strut
\end{minipage} & \begin{minipage}[t]{0.05\columnwidth}\raggedleft
-27.00\strut
\end{minipage} & \begin{minipage}[t]{0.06\columnwidth}\raggedleft
-4.00\strut
\end{minipage} & \begin{minipage}[t]{0.06\columnwidth}\raggedleft
20.00\strut
\end{minipage} & \begin{minipage}[t]{0.07\columnwidth}\raggedleft
14.62\strut
\end{minipage} & \begin{minipage}[t]{0.06\columnwidth}\raggedleft
1640\strut
\end{minipage} & \begin{minipage}[t]{0.07\columnwidth}\raggedleft
1640\strut
\end{minipage} & \begin{minipage}[t]{0.07\columnwidth}\raggedleft
0\strut
\end{minipage} & \begin{minipage}[t]{0.10\columnwidth}\raggedleft
0\strut
\end{minipage}\tabularnewline
\begin{minipage}[t]{0.07\columnwidth}\raggedright
study3\strut
\end{minipage} & \begin{minipage}[t]{0.07\columnwidth}\raggedright
age.60\strut
\end{minipage} & \begin{minipage}[t]{0.04\columnwidth}\raggedleft
-3.03\strut
\end{minipage} & \begin{minipage}[t]{0.05\columnwidth}\raggedleft
-25.00\strut
\end{minipage} & \begin{minipage}[t]{0.06\columnwidth}\raggedleft
-3.00\strut
\end{minipage} & \begin{minipage}[t]{0.06\columnwidth}\raggedleft
20.00\strut
\end{minipage} & \begin{minipage}[t]{0.07\columnwidth}\raggedleft
13.93\strut
\end{minipage} & \begin{minipage}[t]{0.06\columnwidth}\raggedleft
2688\strut
\end{minipage} & \begin{minipage}[t]{0.07\columnwidth}\raggedleft
2688\strut
\end{minipage} & \begin{minipage}[t]{0.07\columnwidth}\raggedleft
0\strut
\end{minipage} & \begin{minipage}[t]{0.10\columnwidth}\raggedleft
0\strut
\end{minipage}\tabularnewline
\begin{minipage}[t]{0.07\columnwidth}\raggedright
study1\strut
\end{minipage} & \begin{minipage}[t]{0.07\columnwidth}\raggedright
endtime\strut
\end{minipage} & \begin{minipage}[t]{0.04\columnwidth}\raggedleft
3.72\strut
\end{minipage} & \begin{minipage}[t]{0.05\columnwidth}\raggedleft
0.20\strut
\end{minipage} & \begin{minipage}[t]{0.06\columnwidth}\raggedleft
3.00\strut
\end{minipage} & \begin{minipage}[t]{0.06\columnwidth}\raggedleft
9.01\strut
\end{minipage} & \begin{minipage}[t]{0.07\columnwidth}\raggedleft
2.76\strut
\end{minipage} & \begin{minipage}[t]{0.06\columnwidth}\raggedleft
2060\strut
\end{minipage} & \begin{minipage}[t]{0.07\columnwidth}\raggedleft
2060\strut
\end{minipage} & \begin{minipage}[t]{0.07\columnwidth}\raggedleft
0\strut
\end{minipage} & \begin{minipage}[t]{0.10\columnwidth}\raggedleft
0\strut
\end{minipage}\tabularnewline
\begin{minipage}[t]{0.07\columnwidth}\raggedright
study2\strut
\end{minipage} & \begin{minipage}[t]{0.07\columnwidth}\raggedright
endtime\strut
\end{minipage} & \begin{minipage}[t]{0.04\columnwidth}\raggedleft
3.93\strut
\end{minipage} & \begin{minipage}[t]{0.05\columnwidth}\raggedleft
0.30\strut
\end{minipage} & \begin{minipage}[t]{0.06\columnwidth}\raggedleft
3.00\strut
\end{minipage} & \begin{minipage}[t]{0.06\columnwidth}\raggedleft
9.47\strut
\end{minipage} & \begin{minipage}[t]{0.07\columnwidth}\raggedleft
2.79\strut
\end{minipage} & \begin{minipage}[t]{0.06\columnwidth}\raggedleft
1640\strut
\end{minipage} & \begin{minipage}[t]{0.07\columnwidth}\raggedleft
1640\strut
\end{minipage} & \begin{minipage}[t]{0.07\columnwidth}\raggedleft
0\strut
\end{minipage} & \begin{minipage}[t]{0.10\columnwidth}\raggedleft
0\strut
\end{minipage}\tabularnewline
\begin{minipage}[t]{0.07\columnwidth}\raggedright
study3\strut
\end{minipage} & \begin{minipage}[t]{0.07\columnwidth}\raggedright
endtime\strut
\end{minipage} & \begin{minipage}[t]{0.04\columnwidth}\raggedleft
3.74\strut
\end{minipage} & \begin{minipage}[t]{0.05\columnwidth}\raggedleft
0.22\strut
\end{minipage} & \begin{minipage}[t]{0.06\columnwidth}\raggedleft
3.00\strut
\end{minipage} & \begin{minipage}[t]{0.06\columnwidth}\raggedleft
8.98\strut
\end{minipage} & \begin{minipage}[t]{0.07\columnwidth}\raggedleft
2.79\strut
\end{minipage} & \begin{minipage}[t]{0.06\columnwidth}\raggedleft
2688\strut
\end{minipage} & \begin{minipage}[t]{0.07\columnwidth}\raggedleft
2688\strut
\end{minipage} & \begin{minipage}[t]{0.07\columnwidth}\raggedleft
0\strut
\end{minipage} & \begin{minipage}[t]{0.10\columnwidth}\raggedleft
0\strut
\end{minipage}\tabularnewline
\begin{minipage}[t]{0.07\columnwidth}\raggedright
study1\strut
\end{minipage} & \begin{minipage}[t]{0.07\columnwidth}\raggedright
starttime\strut
\end{minipage} & \begin{minipage}[t]{0.04\columnwidth}\raggedleft
2.38\strut
\end{minipage} & \begin{minipage}[t]{0.05\columnwidth}\raggedleft
0.00\strut
\end{minipage} & \begin{minipage}[t]{0.06\columnwidth}\raggedleft
2.00\strut
\end{minipage} & \begin{minipage}[t]{0.06\columnwidth}\raggedleft
8.00\strut
\end{minipage} & \begin{minipage}[t]{0.07\columnwidth}\raggedleft
2.64\strut
\end{minipage} & \begin{minipage}[t]{0.06\columnwidth}\raggedleft
2060\strut
\end{minipage} & \begin{minipage}[t]{0.07\columnwidth}\raggedleft
2060\strut
\end{minipage} & \begin{minipage}[t]{0.07\columnwidth}\raggedleft
0\strut
\end{minipage} & \begin{minipage}[t]{0.10\columnwidth}\raggedleft
0\strut
\end{minipage}\tabularnewline
\begin{minipage}[t]{0.07\columnwidth}\raggedright
study2\strut
\end{minipage} & \begin{minipage}[t]{0.07\columnwidth}\raggedright
starttime\strut
\end{minipage} & \begin{minipage}[t]{0.04\columnwidth}\raggedleft
2.57\strut
\end{minipage} & \begin{minipage}[t]{0.05\columnwidth}\raggedleft
0.00\strut
\end{minipage} & \begin{minipage}[t]{0.06\columnwidth}\raggedleft
2.00\strut
\end{minipage} & \begin{minipage}[t]{0.06\columnwidth}\raggedleft
8.00\strut
\end{minipage} & \begin{minipage}[t]{0.07\columnwidth}\raggedleft
2.71\strut
\end{minipage} & \begin{minipage}[t]{0.06\columnwidth}\raggedleft
1640\strut
\end{minipage} & \begin{minipage}[t]{0.07\columnwidth}\raggedleft
1640\strut
\end{minipage} & \begin{minipage}[t]{0.07\columnwidth}\raggedleft
0\strut
\end{minipage} & \begin{minipage}[t]{0.10\columnwidth}\raggedleft
0\strut
\end{minipage}\tabularnewline
\begin{minipage}[t]{0.07\columnwidth}\raggedright
study3\strut
\end{minipage} & \begin{minipage}[t]{0.07\columnwidth}\raggedright
starttime\strut
\end{minipage} & \begin{minipage}[t]{0.04\columnwidth}\raggedleft
2.40\strut
\end{minipage} & \begin{minipage}[t]{0.05\columnwidth}\raggedleft
0.00\strut
\end{minipage} & \begin{minipage}[t]{0.06\columnwidth}\raggedleft
2.00\strut
\end{minipage} & \begin{minipage}[t]{0.06\columnwidth}\raggedleft
8.00\strut
\end{minipage} & \begin{minipage}[t]{0.07\columnwidth}\raggedleft
2.68\strut
\end{minipage} & \begin{minipage}[t]{0.06\columnwidth}\raggedleft
2688\strut
\end{minipage} & \begin{minipage}[t]{0.07\columnwidth}\raggedleft
2688\strut
\end{minipage} & \begin{minipage}[t]{0.07\columnwidth}\raggedleft
0\strut
\end{minipage} & \begin{minipage}[t]{0.10\columnwidth}\raggedleft
0\strut
\end{minipage}\tabularnewline
\begin{minipage}[t]{0.07\columnwidth}\raggedright
Combined\strut
\end{minipage} & \begin{minipage}[t]{0.07\columnwidth}\raggedright
age.60\strut
\end{minipage} & \begin{minipage}[t]{0.04\columnwidth}\raggedleft
-3.32\strut
\end{minipage} & \begin{minipage}[t]{0.05\columnwidth}\raggedleft
-26.16\strut
\end{minipage} & \begin{minipage}[t]{0.06\columnwidth}\raggedleft
-3.26\strut
\end{minipage} & \begin{minipage}[t]{0.06\columnwidth}\raggedleft
20.00\strut
\end{minipage} & \begin{minipage}[t]{0.07\columnwidth}\raggedleft
16216.40\strut
\end{minipage} & \begin{minipage}[t]{0.06\columnwidth}\raggedleft
6388\strut
\end{minipage} & \begin{minipage}[t]{0.07\columnwidth}\raggedleft
6388\strut
\end{minipage} & \begin{minipage}[t]{0.07\columnwidth}\raggedleft
0\strut
\end{minipage} & \begin{minipage}[t]{0.10\columnwidth}\raggedleft
0\strut
\end{minipage}\tabularnewline
\begin{minipage}[t]{0.07\columnwidth}\raggedright
Combined\strut
\end{minipage} & \begin{minipage}[t]{0.07\columnwidth}\raggedright
endtime\strut
\end{minipage} & \begin{minipage}[t]{0.04\columnwidth}\raggedleft
3.78\strut
\end{minipage} & \begin{minipage}[t]{0.05\columnwidth}\raggedleft
0.24\strut
\end{minipage} & \begin{minipage}[t]{0.06\columnwidth}\raggedleft
3.00\strut
\end{minipage} & \begin{minipage}[t]{0.06\columnwidth}\raggedleft
9.11\strut
\end{minipage} & \begin{minipage}[t]{0.07\columnwidth}\raggedleft
618.76\strut
\end{minipage} & \begin{minipage}[t]{0.06\columnwidth}\raggedleft
6388\strut
\end{minipage} & \begin{minipage}[t]{0.07\columnwidth}\raggedleft
6388\strut
\end{minipage} & \begin{minipage}[t]{0.07\columnwidth}\raggedleft
0\strut
\end{minipage} & \begin{minipage}[t]{0.10\columnwidth}\raggedleft
0\strut
\end{minipage}\tabularnewline
\begin{minipage}[t]{0.07\columnwidth}\raggedright
Combined\strut
\end{minipage} & \begin{minipage}[t]{0.07\columnwidth}\raggedright
starttime\strut
\end{minipage} & \begin{minipage}[t]{0.04\columnwidth}\raggedleft
2.44\strut
\end{minipage} & \begin{minipage}[t]{0.05\columnwidth}\raggedleft
0.00\strut
\end{minipage} & \begin{minipage}[t]{0.06\columnwidth}\raggedleft
2.00\strut
\end{minipage} & \begin{minipage}[t]{0.06\columnwidth}\raggedleft
8.00\strut
\end{minipage} & \begin{minipage}[t]{0.07\columnwidth}\raggedleft
571.92\strut
\end{minipage} & \begin{minipage}[t]{0.06\columnwidth}\raggedleft
6388\strut
\end{minipage} & \begin{minipage}[t]{0.07\columnwidth}\raggedleft
6388\strut
\end{minipage} & \begin{minipage}[t]{0.07\columnwidth}\raggedleft
0\strut
\end{minipage} & \begin{minipage}[t]{0.10\columnwidth}\raggedleft
0\strut
\end{minipage}\tabularnewline
\bottomrule
\end{longtable}

\hypertarget{data-quality-and-understanding-covariates}{%
\subsubsection{Data quality and understanding
covariates}\label{data-quality-and-understanding-covariates}}

\hypertarget{missingness-of-covariates}{%
\paragraph{Missingness of covariates}\label{missingness-of-covariates}}

The amount of missing for each covariate is shown below. The table shows
if there are any covariates that are missing for any study. If
covariates are missing, this may lead to warnings or convergence issues
in survival models. Knowing this in advance, the analyst can make
approporiate modifications or transformations to the data.

\begin{Shaded}
\begin{Highlighting}[]
\NormalTok{    dt_missingness <-}\StringTok{ }\NormalTok{dsHelper}\OperatorTok{::}\KeywordTok{dh.anyData}\NormalTok{(}\DataTypeTok{conns =}\NormalTok{ connections,  }
                                           \DataTypeTok{df =} \StringTok{'D'}\NormalTok{, }
                       \DataTypeTok{vars =} \KeywordTok{c}\NormalTok{(}\StringTok{'age.60'}\NormalTok{, }\StringTok{'female'}\NormalTok{, }\StringTok{'starttime'}\NormalTok{, }\StringTok{'endtime'}\NormalTok{)}
\NormalTok{                       ) }

\NormalTok{    knitr}\OperatorTok{::}\KeywordTok{kable}\NormalTok{(dt_missingness)}
\end{Highlighting}
\end{Shaded}

\begin{longtable}[]{@{}llll@{}}
\toprule
variable & study1 & study2 & study3\tabularnewline
\midrule
\endhead
age.60 & TRUE & TRUE & TRUE\tabularnewline
female & TRUE & TRUE & TRUE\tabularnewline
starttime & TRUE & TRUE & TRUE\tabularnewline
endtime & TRUE & TRUE & TRUE\tabularnewline
\bottomrule
\end{longtable}

\hypertarget{variable-types-of-covariates}{%
\subsubsection{Variable types of
covariates}\label{variable-types-of-covariates}}

This summarizes if the covariates are of the same type or class in each
study. Knowing this in advance can prevent errors or convergence issues
in survival models.

\begin{Shaded}
\begin{Highlighting}[]
\NormalTok{  dt_type_information <-}\StringTok{ }\NormalTok{dsHelper}\OperatorTok{::}\KeywordTok{dh.classDiscrepancy}\NormalTok{(}\DataTypeTok{conns =}\NormalTok{ connections, }
                                                       \DataTypeTok{df =} \StringTok{'D'}\NormalTok{,}
                               \DataTypeTok{vars =} \OtherTok{NULL}\NormalTok{)}

\NormalTok{  knitr}\OperatorTok{::}\KeywordTok{kable}\NormalTok{(dt_type_information)}
\end{Highlighting}
\end{Shaded}

\begin{longtable}[]{@{}lllll@{}}
\toprule
variable & discrepancy & study1 & study2 & study3\tabularnewline
\midrule
\endhead
id & no & integer & integer & integer\tabularnewline
study.id & no & integer & integer & integer\tabularnewline
time.id & no & integer & integer & integer\tabularnewline
starttime & no & numeric & numeric & numeric\tabularnewline
endtime & no & numeric & numeric & numeric\tabularnewline
survtime & no & numeric & numeric & numeric\tabularnewline
cens & no & factor & factor & factor\tabularnewline
age.60 & no & numeric & numeric & numeric\tabularnewline
female & no & factor & factor & factor\tabularnewline
noise.56 & no & numeric & numeric & numeric\tabularnewline
pm10.16 & no & numeric & numeric & numeric\tabularnewline
bmi.26 & no & numeric & numeric & numeric\tabularnewline
\bottomrule
\end{longtable}

Finally, once you have finished your analysis, you can disconnect from
the server(s) using the following command:

\begin{Shaded}
\begin{Highlighting}[]
\NormalTok{DSI}\OperatorTok{::}\KeywordTok{datashield.logout}\NormalTok{(}\DataTypeTok{conns =}\NormalTok{ connections)}
\end{Highlighting}
\end{Shaded}

\newpage

\begin{itemize}
\item
  \url{https://github.com/datashield}
\item
  \url{http://www.metafor-project.org}
\item
  \url{https://github.com/neelsoumya/dsSurvival}
\item
  \url{https://github.com/neelsoumya/dsSurvivalClient}
\end{itemize}

\end{document}
